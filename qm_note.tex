\documentclass{ctexart}
\ctexset{scheme=chinese}
\usepackage{amsmath,amsthm,amssymb,physics}

%% 设置页面
% 两面文档; 装订侧 1cm 缩进.
% margin: left=right=2.54cm, bottom=top=3.18cm 
\usepackage{geometry}
\geometry{left=2.54cm,right=2.54cm,top=3.18cm,bottom=3.18cm}
\geometry{twoside, bindingoffset=1cm}
\geometry{b5paper}

% 设置页眉页脚
\usepackage{fancyhdr}
\pagestyle{fancy}
\fancyfoot[C]{~----- \thepage ~-----}

% 设置水印
%\usepackage{background}
% \backgroundsetup{contents=jiamin,color=red!10,scale=15}

% 设置新命令 & environments 
\newcommand{\qed}{\rule[0]{8pt}{8pt}}
% 解、答、注 环境
\newcounter{problemname}
\newenvironment{problem}{\stepcounter{problemname}\par\noindent\textbf{题\arabic{problemname}. }}{\par}
\newenvironment{sol}{\par\noindent\texbf{解. }}{\vspace{2em}} 
\newenvironment{note}{\par\noindent\texbf{题\arabic{problemname}注. }}{\par\vspace{2em}} 

% 定理、引理、证明环境
\newtheorem{theorem}{定理}[section]
\newtheorem{define}{定义}[section] 

% 作者信息
\author{jia} 

% =========
% 正文开始
% =========
\begin{document} 

% 可选,title 
%\maketitle 
\section{力学量}

没有本征值的厄米算符。$\hat{H} = p^2 - x^2$. 处在一个倒转的谐振势中, 因此其会
加速下滑. $\frac{dx}{dt} = \partial_p H = 2p$. $\frac{dp}{dt} = 2x$. 
therefore 
$$
\frac{d}{dt} 
\begin{vmatrix} 
	x \\ p
\end{vmatrix}
= 
\begin{vmatrix } 
	0 & 2 \\ 2 & 0 	
\end{vmatrix }
\begin{vmatrix} 
	x \\ p	
\end{vmatrix}
$$
therefore $\overline{x}, \overline{p} \mapsto \infty$
\subsection{力学量的平均值和方均根} 

\begin{enumerate}
	\item $\overline{A}=\bra{\psi}A\ket{\psi}$
	\item $\sigma_A^2 = \overline{A^2} - (\overline{A})^2$
\end{enumerate}

\begin{theorem}
	\sigma_A^2 \sigma_B^2 \ge 1/4 \langle [A,B]\rangle^2 
\end{theorem}

对于可对易的力学量算符, 有

\begin{enumerate} 
	\item 两个算符对易当且仅当两个算符可以直接对角化(或有同一个完备基)
	\item $[A,B]=0 \ket{\psi}\ne 0, A\ket{\psi}=\lambda\ket{\psi}$.then $B\ket{\psi}$ is also eigenket of $A$. 
	
	\item 上面结论的推论: 在 A 的表象下, B 也是分块对角矩阵. 
	\item $[A,B]=0$ if and only if $A, B$ have common complete eigen-base-ket set. 
	\item If $[A,B]\ne 0$, then they {\em may} have common eigenket, but all the eigenket is {\em not} complete. 
\end{enumerate}

一维无限深势井 (对称). 则哈密顿算符和Parity算符对易. i.e. $\Pi \ket{a}$ state
also has an energy of $E_a$. That is, $\Pi\ket{a}\propto \ket{a}$. 

Take simple harmonic oscillator as example, $[H,\Pi]=0$ , i.e. $\Pi\ket{n}
=c\ket{n}$. 

\subsection{力学量算符完全集}

Complete Set of Commutating Operators (CSCO): a set of commutating operators that can eliminate degenracy of eigenkets of each eigenvalue. 

For a given physical system, it may have multiple set of CSCO. The exact choice
of CSCO depends on the problem you want to examine. For example, the {\em
coupled} and {\em decoupled} representation of angular momentumn. 

Some example of CSCO:
\begin{enumerate}
	\item $p_x, p_y, p_z$ 
	\item $x, y, z$ 
	\item $(p_x, y, z)$
	\item $(p^2 / 2m, L^2, L_z)$
	\item $(L^2, L_z, S^2, S_z)$ and $(J^2, J_z, L^2, S^2)$
\end{enumerate}
The use of the 3rd CSCO is: when studying 2D material, we may have to use $k$
representation and $\vec{x}$ representation simutaniously for convinience. 

坐标表象和动量表象的关系看 Cohen 的书. 

\subsection{测量的相容性问题}

测量的相容性问题等价于算符的可对易性质. 

\begin{define}[相容性] 
	如果有两个力学量, 相互对易. 则在测量时, 我们可以同时确定两个力学量的测
	量值. 用数学表示, 则是 
	$$
	A \ket{a_n, b_p, i} = a_n \ket{a_n, b_p, i} 
	$$
	$$
	B \ket{a_n, b_p, i} = b_p \ket{a_n, b_p, i} 
	$$
	此处, $a_n, b_p, i$ 分别是 $A, B$ 的本征值和共同简并度. 
	并且, 如果 $A, B$ 对易, 则先测量 $A$, 再测量 $B$; 和先测量 $B$, 再测量
	$A$ 的结果是一样的. 
\end{define}

解释一下. 先侧 $A$, then $B$. 设 
$$
\ket{\psi} = \sum_{n,p,i}^{\infty}  c_{n,p,i}\ket{a_n, b_p, i} 
$$

\section{量子化规则}

这一节研究
\begin{itemize} 
    \item 如何赋予力学量算子? 
    \item 如何描述力学量的时间演化? 
\end{itemize}
解决这两个问题, 我们就进入了量子力学. 这个过程是一个无厘头的过程. 有两套量子化的
方法: 1) 正则量子化. 由薛定谔和海森堡给出. 2) 路径积分量子化, 由费曼提出. 

正则量子化: 由经典力学语言转化到哈密顿力学. 即 Poisson Bracket 和 对易子之间的关
系. 

路径积分量子化基于拉格朗日量(在Lorentz变换下不变). 

我们采用正则量子化. 其又分成两种方法. 第一种: 通过 Poisson Brackets 完成, 即通过
将 Poisson 括号升级为量子力学中的对易子. 第二种: 经典力学中的正则变换升级为量子
力学中的 Unitary 对称变换的等价性完成. 
\footnote{其实这两种方法完全等价. Sakurai 的方法是采用正则变换的方法}
正则变换强调了对称变换的问题. 而空间平移算符和空间转动算符直接导致了动量和角动量
. 

我们两种方法都讲. 

\subsection{基于泊松括号的量子化规则} 

如果考虑处于一个标量场中的无自旋粒子, 规定与 $(x,y,z)$ 对应的算子为 $(X, Y, Z)$;
规定与 $p_x,p_y,p_z)$ 对应的算符为 $P=-i\hbar \nalba$. 由此推出, 
\[
\forall A, A=A(r,p,t) \mapsto \hat{A} = \hat{A}(R,P,t)
\]
升级到量子力学后, $\vec{R}\cdot \vec{P}\ne \vec{P}\cdot \vec{R}$. 为了解决这个问
题, 将这个东西变成 $\frac{1}{2} (\vec{r}\codt \vec{p} + \vec{p}\codt \vec{r})$ . 

对力学量算符选择的原则:
\begin{itemize} 
    \item 对称性
    \item 无论如何选择, 我们的选择都是主观的. 因此需要和实验对比. 只要能解释实验,
	就说明我们的选择是正确的. 
    \item 在某种极限下, 量子力学将回归经典力学, 二者给出相同的语言. 
\end{itemize}

量子力学中, $[R_i,P_j]=i\hbar\delta_{ij}$. 这告诉我们经典力学和量子力学对易关系
的对应为 
\[
    \{,\} \mapsto \frac{1}{i\hbar }[,]
\]

从经典到量子分下面几步: 
1) $[Q_i, P_j]=i\hbar\delta_{ij}$. 
2) 哈密顿方程变成薛定谔方程. 哈密顿方程为 
\[
    \frac{dq_i}{dt} = \{q_i, H\} = \frac{\partial H}{\partial p_i}
\]
\[
    \frac{dp_i}{dt} = \{p_i,H\} = -\frac{\partial H}{\partial q_i} = -
    \frac{\partial H}{\partial t}
\]
海森堡picture为 
\[
\frac{d\Omega}{dt} = \frac{\partial \Omega}{\partial t} + \frac{1}{i\hbar }
[\Omega, H]
\]
\footnote{若 $\Omega$ 和时间无关, 且 $[\Omega,H]=0$, 则 $\Omega$ 不随着时间演化.}

Shrodinger的假设为: $\forall \ket{\psi(t)}, H(t): i\hbar \frac{d}{dt}
\ket{\psi(t)} = H(t)\ket{\psi(t)}$. 
实验中有意义的是, 力学量 $\Omega$ 在时间 $t$ 的期望值. 期望写作 
\[
    \bra{\psi(t)}\Omega\ket{\psi(t)} 
\]
\[
\frac{d}{dt}\overline{\Omega} = \bra{\psi}\frac{\partial \Omega}{\partial t}
\ket{\psi} 
+ [\frac{d}{dt}\ket{\psi}]\Omega\ket{\psi} 
+ \bra{\psi}\Omega[\frac{d}{dt}\ket{\psi}]
\]
\[
\Rightarrow \frac{d}{dt}\overline{\Omega} = \bra{\psi}\frac{\partial\Omega}
{\partial t} \ket{\psi} + \frac{1}{i\hbar }\bra{\psi}[\Omega, H] \ket{\psi} 
\]
经典力学中, 泊松括号和正则坐标选择无关. 

\subsection{对称变换:连续} 

所谓连续对称变换, 指的是对称变换可以用一个连续的参数描述的对称变换. 所以反射变换
/格点平移操作 不是连续性的对称变换. 

此处, 对称性是力学量在变换下所具有的不变性. 

\begin{theorem}[Nother's Law]
    一个对称不变性对应一个守恒量. 
\end{theorem}

例如, 若 $L(q+dq, v, t)= L(q,v,t)$, 则根据拉格朗日方程可知, 正则动量
$\frac{\partial}{\partial v}L=\frac{d}{dt}P=0$. 即正则动量守恒. 

量子力学中的连续对称变换其实是经典力学中正则变换的推广. 因此将诺特定理推广到量子
体系: 空间平移/旋转 对应动量和角动量. 

平移操作. 
\[
T(a): T(a)\psi(x) \mapsto \eta(x): eta(x+a) = \psi(x)
\]
\[
\begin{align} 
    \eta(x) &= \psi(x-a) \\
	    &= \sum_{n=1}^{\infty} \frac{1}{n!} (\frac{d}{dx})^n \psi(x) \\
	    &= \sum_{n=0}^{\infty}\frac{1}{n!}\Bigl[ (-i) a(-\frac{id}{dx})\Bigr]
	    ^n\psi(x) \\
	    &= \exp(-ia\codt p) \psi(x) 
\end{align}
\]
因此, 平移算子
 \[
T(a) \equiv \exp(-i \vec{a}\cdot \vec{p}) 
\]
如果 $a \ll 1$, then $\eta(x)=(1 - ia\cdot p)\psi(x)$. 因此 
\[
T(a\ll 1) = 1 - i \vec{a}\cdot \vec{p} 
\]
称作无穷小平移算子. $p$ 是无穷小平移的生成元. 

定轴转动: $L=\frac{1}{2}M\codt\theta^2$. 
\[
R_n(d\theta): \psi(\vec{r}) \mapsto \psi(\vec{r}-d \vec{r}); d \vec{r}=\vec{n}
\times \vec{r} d\theta 
\]
\[
    \begin{align*} 
	\psi(\vec{r}- d \vec{r}) &= \psi(\vec{r})-\nabla \psi(\vec{r}) \cdot d \vec{r}
	\\
	&= \psi(\vec{r}) - \nabla \psi(\vec{r})(d\theta \vec{n}\times \vec{r})\\
	&= \psi(\vec{r}) - d\theta \vec{n}\cdot (\vec{r}\times\nabla\psi(r)) \\
	&= [1 - i\vec{n}\codt(\vec{r}\times(-i\nabla))]\psi(\vec{r})
    \end{align*}
\]
因此, 无穷小转动算符为 
\[
R_n(d\theta) = 1 - id\theta \vec{n}\cdot(\vec{r}\times (-i\nabla))
\]
注意, $\vec{n}\cdot[\vec{r}\times(-i\nabla)]=L_n$. 
现在计算 $R_n(\theta+d\theta)$. 
\begin{align*} 
    R_n(\theta+d\theta) &= R_n(\theta)R_n(d\theta) 
\end{align*}

\begin{problem}[作业]
    利用角动量是无穷小转动的生成元这一观念, 求出球坐标下 $L_x, L_y, L_z$ 的形式.
\end{problem}
\begin{sol} 
    $L_z$ : $R_z(\epsilon\ll 1)$. 
    $L_x$ : 是绕着 x 旋转的无穷小生成元。考虑 $R_x (\epsilon)$. 
    $x = r\sin\theta\cos \phi , y = r \sin \theta \sin \phi $. 
    After rotation, $r'\mapsto r, \phi \mapsto \phi+d \phi, \theta' \mapsto
    \theta + d \theta$.  此时, 由于绕着 x 轴旋转, 因此 x 坐标不变:
    \[
	x' \mapsto r\sin( \theta + d \theta )\cos( \phi + d \phi )  = x
    \]
    泰勒展开到一阶,我们得到
    \[
	(\sin \theta + \cos \theta d \theta )(\cos \phi - \sin \phi d \phi )
	= \sin \theta \cos \phi 
    \]
    保留一阶微分项,则
    $d \theta = - \frac{\sin \theta \sin \phi }{\cos \theta \cos \phi }d \phi
    $. 
    
    变换后,新坐标和老坐标之间的关系为
    \[
    \begin{cases}
	z' &= \cos \epsilon z + \sin \epsilon y \\
	y' &= \cos \epsilon y - \sin \epsilon z
    \end{cases}
    \]
    这个式子的得出需要利用 $x,y,z$ 的轮换对称性. 然后利用绕 $z$ 旋转时 $x,y$ 
    的变换规则写出上面两个式子. 利用 $z', z$ 之间的关系,得到
    \[
	\cos( \theta + d \theta ) 
	= \cos \epsilon \cos \theta + \sin \epsilon \sin \theta \sin \phi 
    \]
    \[
    \Rightarrow d \theta = - \epsilon \sin \phi 
    \]
    \[
    \Rightarrow d \phi = \frac{\cos \theta \cos \phi }{\sin \theta } \epsilon 
    \]
    \[
    \psi(r) \mapsto g(r), 
    \eta (r, \theta, \phi ) = \psi (r, \theta-d \theta, \phi-d \phi)
    = \psi(r, \theta+\sin \phi \epsilon , \phi + \frac{\cos \theta \cos \phi }
    {\sin \theta }\epsilon )
    \]
    \[
    = \psi+\sin \phi \epsilon \frac{\partial \psi }{\partial \theta } 
    + \frac{\cos \theta \cos \phi }{\sin \theta }\epsilon \frac{\partial \psi }
    {\partial \phi } 
= [1 - i \epsilon L_x ] \psi 
\]
   i.e. 
   \[
   L_x = i\sin \phi \frac{\partial  }{\partial \theta } + 
   i\cos \phi \cot \theta \frac{\partial  }{\partial \phi }  
   \]
   \qed
\end{sol} 

\subsection{平移对称操作的形式理论} 

$T (a) \ket{r} = \ket{r+a}$ 是平移操作定义. 于是 $\forall \psi$, 考虑 
\[
\bra{r}T(a) \ket{\psi} = 
\int dr' \bra{r}T(a)\ket{r'}\braket{r'}{\psi}
\]
这等于 
\[
\int dr' \delta (r-r'-a) \psi (r') = \psi (r-a) 
\]

平移操作的性质:
\begin{itemize}
    \item 性质1. $\bigl[ T(dx) \bigr]^\dag T(dx) = I$. This is required by the
	normalization of prob. 
    \item 性质2. It is an abellian group. i.e. $\forall a, b: T(a)T(b)=T(b)T(a)=T(a+b)$ 
    \item 性质三:$\lim_{dx\mapsto 0}T(dx)=I$
\end{itemize} 

展开 T(a): $T(dx)=I-i\vec{k}\cdot d \vec{x}$ . 容易知道 $T^\dag(dx) = I + i
\vec{k}\cdot d \vec{x} = T(-dx) = T^{-1}(dx)$. 
\marginpar{\footnotesize 为什么这么展开?}
\[
  \vec{k} = i \frac{dT}{dx}\Big|_{x_0=0} = i \nabla T \big|_{x_0=0}
\]
\[
    T^\dag (dx) T(dx) = I = [ I + i \vec{k}\cdot d \vec{x} ] 
[ I - i \vec{k} \cdot d \vec{x} ] 
= I + i \vec{dx} \cdot [ k^\dag - k ]
\]
因此 $k^\dag = k$. 

\[
T(a) = \exp(-i \vec{k} \cdot \vec{a}) 
\]

What is the commutator of $[R,k]$? 
$\forall \vec{r}: T(a)R\ket{\vec{r}}- RT(a)\ket{\vec{r}} = -a T(a)\ket{\vec{r}}
$
 
这里先计算一个结论: 
\[
f( \lambda ) = e^{ \lambda A} B e^{- \lambda A}
\]
\[
\frac{d}{d \lambda }f( \lambda ) \Bigr|_{\lambda =0} = 
e^{\lambda A}[A,B]e^{- \lambda A} = [A, B] 
\]
\[
    \frac{d^2 }{d \lambda ^2 }f( \lambda )\Bigr|_{\lambda =0} = [A, [A,B]
\]
\[
\therefore e^A B e^{-A} = \sum_{n=0}^{\infty} \frac{1}{n!} f^{(n)}( \lambda =0) 
= \sum_{n=1}^{\infty} \frac{1}{ n!} [C_A ]^n B 
= e^{C_A} B 
\]
\begin{problem}[Homework]
    证明 $e^A B e^A = e^{S_A}B$. $S_A B = \{A,B\}$. 
\end{problem}

\begin{theorem}[CBH equation]
    if $[[A,B],A] = [[A,B],B]$, \text{then  
    $e^A e^B = e^{A+B} e^{\frac{1}{2}[A,B] }$. 
\end{theorem}
\begin{proof}
    $F(t) = e^{At}e^{Bt}$, 
    \[
	\frac{d}{dt} e^{At} (A+B) e^{Bt} = [A + e^{At} B e^{-At} ] e^{At} e^{Bt}
    \]
    \[
	= [A+B+t[A,B] ] e^{At}e^{Bt} 
    \]
    
    Homework
    \[
	e^{At} e^{Bt} = \exp[(A+B)t + \frac{1}{2} t^2 [A,B]] 
    \]
\end{proof}

Calculate $T(a)R T^{-1}(a) = \exp[-i \vec{k}\cdot \vec{r}] \exp[i \vec{k}\cdot
\vec{a} ]$. This equals to $R+[-i \vec{k}\cdot \vec{a}, \vec{R}] + \frac{1}{2}
[-i \vec{k} \cdot \vec{a}, [ - i \vec{k} \vec{a}, R]]$
therefore $T(a)R T^{-1}(a) = R - a$. 
so $-a = [-i \vec{k} \cdot \vec{a}, \vec{R}] \Rightarrow [R_i ,K_j ]=i
\delta_{ij}$ 

We can also conclude that $[K_i ,K_j ]=0$. This is because 
$[T(dy),T(dx)=0$ ~-- i.e. $[1-i K_y dy, 1 -i K_x dx]=0$. 

Therefore we rename $K$ as momentumn operator, which is the genreator of
translation operation. 

\subsubsection{Deriviation of Hamilton Equation Using Generator} 

Define  $S = S( q^i, q^f, t^i, t^f)$. Then 

\[
\begin{align*}
    \delta S 
    &= \int_{t^i}^{t_f} 
    \Bigl[ \frac{\partial L}{\partial q} \delta q 
    + \frac{\partial L}{\partial \dot{q}} \delta \dot{q}\Bigr] dt \\
    &= \int \biggl[ \frac{\partial L}{\partial q} \delta q 
    + \frac{d}{dt} \frac{\partial L}{\partial \dot{q}} \biggr]
\end{align*}
\]

\subsubsection{Canonical transformation and Generator}

若在拉格朗日力学下, 写出粒子的拉格朗日量. $L(q, \dot{q}, t)$. If we use another
coordinate to rewrite L as $L(Q, Q, t)$. But the Euler-Lagrange eq. is
invariant, so $Q$ is time function of $q$ : $q \mapsto Q(q,t)$. So we can
transition to Hamiltion mechanics. 哈密顿力学允许更广泛 的变换. 哈密顿力学允许
新坐标作为旧坐标/动量的函数; 新动量作为旧坐标/动量的函数. 

并非所有这种变换可以保持哈密顿方程不变. 要 $(q,p,H)\mapsto (Q,P,H)$, 要求新系统
仍旧保持正则方程性质不变, 问需要对变换加入什么限制? 

如果正则方程满足, 等价于作用量的变分等于0. 这意味着 
$\delta [\int PdQ - H' dt] =0$. $H'$ is the new Hamiltionian. \\
Old: $\delta \int [pdq - Hdt]=0$. 
If $pdq - Hdt = PdQ - H'dt + dF$. Here the $F$ funciton is called the genrator
(生成函数). 

我们可以把 $dF = pdq - PdQ + (H'-H) dt$. i.e. $F=F(q,Q,t)$. 此时,
$p = \frac{\partial F}{\partial q} , P = \frac{\partial F}{\partial Q}, H' + H
+ \frac{\partial F}{\partial t} $ 

正则变换是保泊松括号的. 

我们用的生成函数是 $F+PQ$. 
$d(F+PQ) = pdq + QdP + (H' -H) dt$, $p = \frac{\partial \Phi }{\partial q} $, 
$Q = \frac{\partial \Phi }{\partial P} $, $H' = H + \frac{\partial \Phi }{\partial
t} $












\end{document}
