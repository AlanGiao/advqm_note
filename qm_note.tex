\documentclass{ctexart}
\ctexset{scheme=chinese}
\usepackage{amsmath,amsthm,amssymb,physics}

%% 设置页面
% 两面文档; 装订侧 1cm 缩进.
% margin: left=right=2.54cm, bottom=top=3.18cm 
\usepackage{geometry}
\geometry{left=2.54cm,right=2.54cm,top=3.18cm,bottom=3.18cm}
\geometry{twoside, bindingoffset=1cm}
\geometry{b5paper}

% 设置页眉页脚
\usepackage{fancyhdr}
\pagestyle{fancy}
\fancyfoot[C]{~----- \thepage ~-----}

% 设置水印
%\usepackage{background}
% \backgroundsetup{contents=jiamin,color=red!10,scale=15}

% 设置新命令 & environments 
\newcommand{\qed}{\rule[0]{8pt}{8pt}}
% 解、答、注 环境
\newcounter{problemname}
\newenvironment{problem}{\stepcounter{problemname}\par\noindent\textbf{题\arabic{problemname}. }}{\par}
\newenvironment{sol}{\par\noindent\texbf{解. }}{\vspace{2em}} 
\newenvironment{note}{\par\noindent\texbf{题\arabic{problemname}注. }}{\par\vspace{2em}} 

% 定理、引理、证明环境
\newtheorem{theorem}{定理}[section]
\newtheorem{define}{定义}[section] 

% 作者信息
\author{jia} 

% =========
% 正文开始
% =========
\begin{document} 

% 可选,title 
%\maketitle 
\section{力学量}

没有本征值的厄米算符。$\hat{H} = p^2 - x^2$. 处在一个倒转的谐振势中, 因此其会
加速下滑. $\frac{dx}{dt} = \partial_p H = 2p$. $\frac{dp}{dt} = 2x$. 
therefore 
$$
\frac{d}{dt} 
\begin{vmatrix} 
	x \\ p
\end{vmatrix}
= 
\begin{vmatrix } 
	0 & 2 \\ 2 & 0 	
\end{vmatrix }
\begin{vmatrix} 
	x \\ p	
\end{vmatrix}
$$
therefore $\overline{x}, \overline{p} \mapsto \infty$
\subsection{力学量的平均值和方均根} 

\begin{enumerate}
	\item $\overline{A}=\bra{\psi}A\ket{\psi}$
	\item $\sigma_A^2 = \overline{A^2} - (\overline{A})^2$
\end{enumerate}

\begin{theorem}
	\sigma_A^2 \sigma_B^2 \ge 1/4 \langle [A,B]\rangle^2 
\end{theorem}

对于可对易的力学量算符, 有

\begin{enumerate} 
	\item 两个算符对易当且仅当两个算符可以直接对角化(或有同一个完备基)
	\item $[A,B]=0 \ket{\psi}\ne 0, A\ket{\psi}=\lambda\ket{\psi}$.then $B\ket{\psi}$ is also eigenket of $A$. 
	
	\item 上面结论的推论: 在 A 的表象下, B 也是分块对角矩阵. 
	\item $[A,B]=0$ if and only if $A, B$ have common complete eigen-base-ket set. 
	\item If $[A,B]\ne 0$, then they {\em may} have common eigenket, but all the eigenket is {\em not} complete. 
\end{enumerate}

一维无限深势井 (对称). 则哈密顿算符和Parity算符对易. i.e. $\Pi \ket{a}$ state
also has an energy of $E_a$. That is, $\Pi\ket{a}\propto \ket{a}$. 

Take simple harmonic oscillator as example, $[H,\Pi]=0$ , i.e. $\Pi\ket{n}
=c\ket{n}$. 

\subsection{力学量算符完全集}

Complete Set of Commutating Operators (CSCO): a set of commutating operators that can eliminate degenracy of eigenkets of each eigenvalue. 

For a given physical system, it may have multiple set of CSCO. The exact choice
of CSCO depends on the problem you want to examine. For example, the {\em
coupled} and {\em decoupled} representation of angular momentumn. 

Some example of CSCO:
\begin{enumerate}
	\item $p_x, p_y, p_z$ 
	\item $x, y, z$ 
	\item $(p_x, y, z)$
	\item $(p^2 / 2m, L^2, L_z)$
	\item $(L^2, L_z, S^2, S_z)$ and $(J^2, J_z, L^2, S^2)$
\end{enumerate}
The use of the 3rd CSCO is: when studying 2D material, we may have to use $k$
representation and $\vec{x}$ representation simutaniously for convinience. 

坐标表象和动量表象的关系看 Cohen 的书. 

\subsection{测量的相容性问题}

测量的相容性问题等价于算符的可对易性质. 

\begin{define}[相容性] 
	如果有两个力学量, 相互对易. 则在测量时, 我们可以同时确定两个力学量的测
	量值. 用数学表示, 则是 
	$$
	A \ket{a_n, b_p, i} = a_n \ket{a_n, b_p, i} 
	$$
	$$
	B \ket{a_n, b_p, i} = b_p \ket{a_n, b_p, i} 
	$$
	此处, $a_n, b_p, i$ 分别是 $A, B$ 的本征值和共同简并度. 
	并且, 如果 $A, B$ 对易, 则先测量 $A$, 再测量 $B$; 和先测量 $B$, 再测量
	$A$ 的结果是一样的. 
\end{define}

解释一下. 先侧 $A$, then $B$. 设 
$$
\ket{\psi} = \sum_{n,p,i}^{\infty}  c_{n,p,i}\ket{a_n, b_p, i} 
$$

\section{量子化规则}

这一节研究
\begin{itemize} 
    \item 如何赋予力学量算子? 
    \item 如何描述力学量的时间演化? 
\end{itemize}
解决这两个问题, 我们就进入了量子力学. 这个过程是一个无厘头的过程. 有两套量子化的
方法: 1) 正则量子化. 由薛定谔和海森堡给出. 2) 路径积分量子化, 由费曼提出. 

正则量子化: 由经典力学语言转化到哈密顿力学. 即 Poisson Bracket 和 对易子之间的关
系. 

路径积分量子化基于拉格朗日量(在Lorentz变换下不变). 

我们采用正则量子化. 其又分成两种方法. 第一种: 通过 Poisson Brackets 完成, 即通过
将 Poisson 括号升级为量子力学中的对易子. 第二种: 经典力学中的正则变换升级为量子
力学中的 Unitary 对称变换的等价性完成. 
\footnote{其实这两种方法完全等价. Sakurai 的方法是采用正则变换的方法}
正则变换强调了对称变换的问题. 而空间平移算符和空间转动算符直接导致了动量和角动量
. 

我们两种方法都讲. 

\subsection{基于泊松括号的量子化规则} 

如果考虑处于一个标量场中的无自旋粒子, 规定与 $(x,y,z)$ 对应的算子为 $(X, Y, Z)$;
规定与 $p_x,p_y,p_z)$ 对应的算符为 $P=-i\hbar \nalba$. 由此推出, 
\[
\forall A, A=A(r,p,t) \mapsto \hat{A} = \hat{A}(R,P,t)
\]
升级到量子力学后, $\vec{R}\cdot \vec{P}\ne \vec{P}\cdot \vec{R}$. 为了解决这个问
题, 将这个东西变成 $\frac{1}{2} (\vec{r}\codt \vec{p} + \vec{p}\codt \vec{r})$ . 

对力学量算符选择的原则:
\begin{itemize} 
    \item 对称性
    \item 无论如何选择, 我们的选择都是主观的. 因此需要和实验对比. 只要能解释实验,
	就说明我们的选择是正确的. 
    \item 在某种极限下, 量子力学将回归经典力学, 二者给出相同的语言. 
\end{itemize}

量子力学中, $[R_i,P_j]=i\hbar\delta_{ij}$. 这告诉我们经典力学和量子力学对易关系
的对应为 
\[
    \{,\} \mapsto \frac{1}{i\hbar }[,]
\]

从经典到量子分下面几步: 
1) $[Q_i, P_j]=i\hbar\delta_{ij}$. 
2) 哈密顿方程变成薛定谔方程. 哈密顿方程为 
\[
    \frac{dq_i}{dt} = \{q_i, H\} = \frac{\partial H}{\partial p_i}
\]
\[
    \frac{dp_i}{dt} = \{p_i,H\} = -\frac{\partial H}{\partial q_i} = -
    \frac{\partial H}{\partial t}
\]
海森堡picture为 
\[
\frac{d\Omega}{dt} = \frac{\partial \Omega}{\partial t} + \frac{1}{i\hbar }
[\Omega, H]
\]
\footnote{若 $\Omega$ 和时间无关, 且 $[\Omega,H]=0$, 则 $\Omega$ 不随着时间演化.}

Shrodinger的假设为: $\forall \ket{\psi(t)}, H(t): i\hbar \frac{d}{dt}
\ket{\psi(t)} = H(t)\ket{\psi(t)}$. 
实验中有意义的是, 力学量 $\Omega$ 在时间 $t$ 的期望值. 期望写作 
\[
    \bra{\psi(t)}\Omega\ket{\psi(t)} 
\]
\[
\frac{d}{dt}\overline{\Omega} = \bra{\psi}\frac{\partial \Omega}{\partial t}
\ket{\psi} 
+ [\frac{d}{dt}\ket{\psi}]\Omega\ket{\psi} 
+ \bra{\psi}\Omega[\frac{d}{dt}\ket{\psi}]
\]
\[
\Rightarrow \frac{d}{dt}\overline{\Omega} = \bra{\psi}\frac{\partial\Omega}
{\partial t} \ket{\psi} + \frac{1}{i\hbar }\bra{\psi}[\Omega, H] \ket{\psi} 
\]
经典力学中, 泊松括号和正则坐标选择无关. 

\subsection{对称变换:连续} 

所谓连续对称变换, 指的是对称变换可以用一个连续的参数描述的对称变换. 所以反射变换
/格点平移操作 不是连续性的对称变换. 

此处, 对称性是力学量在变换下所具有的不变性. 

\begin{theorem}[Nother's Law]
    一个对称不变性对应一个守恒量. 
\end{theorem}

例如, 若 $L(q+dq, v, t)= L(q,v,t)$, 则根据拉格朗日方程可知, 正则动量
$\frac{\partial}{\partial v}L=\frac{d}{dt}P=0$. 即正则动量守恒. 

量子力学中的连续对称变换其实是经典力学中正则变换的推广. 因此将诺特定理推广到量子
体系: 空间平移/旋转 对应动量和角动量. 

平移操作. 
\[
T(a): T(a)\psi(x) \mapsto \eta(x): eta(x+a) = \psi(x)
\]
\[
\begin{align} 
    \eta(x) &= \psi(x-a) \\
	    &= \sum_{n=1}^{\infty} \frac{1}{n!} (\frac{d}{dx})^n \psi(x) \\
	    &= \sum_{n=0}^{\infty}\frac{1}{n!}\Bigl[ (-i) a(-\frac{id}{dx})\Bigr]
	    ^n\psi(x) \\
	    &= \exp(-ia\codt p) \psi(x) 
\end{align}
\]
因此, 平移算子
 \[
T(a) \equiv \exp(-i \vec{a}\cdot \vec{p}) 
\]
如果 $a \ll 1$, then $\eta(x)=(1 - ia\cdot p)\psi(x)$. 因此 
\[
T(a\ll 1) = 1 - i \vec{a}\cdot \vec{p} 
\]
称作无穷小平移算子. $p$ 是无穷小平移的生成元. 

定轴转动: $L=\frac{1}{2}M\codt\theta^2$. 
\[
R_n(d\theta): \psi(\vec{r}) \mapsto \psi(\vec{r}-d \vec{r}); d \vec{r}=\vec{n}
\times \vec{r} d\theta 
\]
\[
    \begin{align*} 
	\psi(\vec{r}- d \vec{r}) &= \psi(\vec{r})-\nabla \psi(\vec{r}) \cdot d \vec{r}
	\\
	&= \psi(\vec{r}) - \nabla \psi(\vec{r})(d\theta \vec{n}\times \vec{r})\\
	&= \psi(\vec{r}) - d\theta \vec{n}\cdot (\vec{r}\times\nabla\psi(r)) \\
	&= [1 - i\vec{n}\codt(\vec{r}\times(-i\nabla))]\psi(\vec{r})
    \end{align*}
\]
因此, 无穷小转动算符为 
\[
R_n(d\theta) = 1 - id\theta \vec{n}\cdot(\vec{r}\times (-i\nabla))
\]
注意, $\vec{n}\cdot[\vec{r}\times(-i\nabla)]=L_n$. 
现在计算 $R_n(\theta+d\theta)$. 
\begin{align*} 
    R_n(\theta+d\theta) &= R_n(\theta)R_n(d\theta) 
\end{align*}

\begin{problem}[作业]
    利用角动量是无穷小转动的生成元这一观念, 求出球坐标下 $L_x, L_y, L_z$ 的形式.
\end{problem}
\begin{sol} 
    $L_z$ : $R_z(\epsilon\ll 1)$. 
    $L_x$ : 是绕着 x 旋转的无穷小生成元。考虑 $R_x (\epsilon)$. 
    $x = r\sin\theta\cos \phi , y = r \sin \theta \sin \phi $. 
    After rotation, $r'\mapsto r, \phi \mapsto \phi+d \phi, \theta' \mapsto
    \theta + d \theta$.  此时, 由于绕着 x 轴旋转, 因此 x 坐标不变:
    \[
	x' \mapsto r\sin( \theta + d \theta )\cos( \phi + d \phi )  = x
    \]
    泰勒展开到一阶,我们得到
    \[
	(\sin \theta + \cos \theta d \theta )(\cos \phi - \sin \phi d \phi )
	= \sin \theta \cos \phi 
    \]
    保留一阶微分项,则
    $d \theta = - \frac{\sin \theta \sin \phi }{\cos \theta \cos \phi }d \phi
    $. 
    
    变换后,新坐标和老坐标之间的关系为
    \[
    \begin{cases}
	z' &= \cos \epsilon z + \sin \epsilon y \\
	y' &= \cos \epsilon y - \sin \epsilon z
    \end{cases}
    \]
    这个式子的得出需要利用 $x,y,z$ 的轮换对称性. 然后利用绕 $z$ 旋转时 $x,y$ 
    的变换规则写出上面两个式子. 利用 $z', z$ 之间的关系,得到
    \[
	\cos( \theta + d \theta ) 
	= \cos \epsilon \cos \theta + \sin \epsilon \sin \theta \sin \phi 
    \]
    \[
    \Rightarrow d \theta = - \epsilon \sin \phi 
    \]
    \[
    \Rightarrow d \phi = \frac{\cos \theta \cos \phi }{\sin \theta } \epsilon 
    \]
    \[
    \psi(r) \mapsto g(r), 
    \eta (r, \theta, \phi ) = \psi (r, \theta-d \theta, \phi-d \phi)
    = \psi(r, \theta+\sin \phi \epsilon , \phi + \frac{\cos \theta \cos \phi }
    {\sin \theta }\epsilon )
    \]
    \[
    = \psi+\sin \phi \epsilon \frac{\partial \psi }{\partial \theta } 
    + \frac{\cos \theta \cos \phi }{\sin \theta }\epsilon \frac{\partial \psi }
    {\partial \phi } 
= [1 - i \epsilon L_x ] \psi 
\]
   i.e. 
   \[
   L_x = i\sin \phi \frac{\partial  }{\partial \theta } + 
   i\cos \phi \cot \theta \frac{\partial  }{\partial \phi }  
   \]
   \qed
\end{sol} 

\subsection{平移操作的形式理论}

$T(a) \ket{r} = \ket{r+a}$ 是平移操作的定义. $\forall \psi,$ 有
\[
    \begin{align*}
	\bra{r}T(a)\ket{\psi} &= \int dr' \bra{r}T(a)\ket{r'}\braket{r'}{\psi}
	\\
			      &= \int dr' \delta (r-r'-a) \psi (r') = \psi (r-a)
    \end{align*}
\]

平移操作的性质:
\begin{enumerate}
    \item $[T(dx)]^\dag T(dx) = I$. 这是由概率的归一化要求的. 
    \item 平移操作是一个阿贝尔群. 即 $\forall a,b: T(a)T(b)=T(b)T(a)=T(a+b)$. 
    \item $\lim_{dx \to 0}T(dx)=I $
\end{enumerate}

展开 $T(a)$ : 在 0 附近展开得到 $T(dx) = I - \vec{ik}\cdot d \vec{x}$. 容易知道 
$T^\dag = I + i \vec{k} \cdot d \vec{x} = T^{-1} (dx)$. 
\marginpar{\footnotesize 为何这样展开?} 
因此 
\[
\vec{k} = i \frac{dT}{dx}\Big|_{x_0=0} = i \nabla T|_{x_0=0}
\]
所以
\[
\begin{align*}
    T^\dag (dx) T(dx) &= I \\
		      &= [ I + i \vec{k} \cdot d \vec{x} ] [I - i \vec{k} \cdot
		      d \vec{x} ] \\
		      &= I + i d \vec{x} \cdot [ k^\dag - k] 
\end{align*}
\]
$\therefore k^\dag = k$. 

还可以知道 $T(a) = \exp[- i \vec{k} \cdot \vec{a}]$ . 

这里插入一个点, $[T(a), R] \ket{r} = T(a) R \ket{r} - R T(a) \ket{r} = -a T(a)
\ket{r} $ . 

那么 $[R,k]$ 等于多少呢? 下面先计算一个结论. 

\begin{problem}
    证明  $e^A B e^{- \lambda A}= $
\end{problem}
\begin{sol}
    先计算 
    \[
	f( \lambda ) = e^{ \lambda A } B e^{- \lambda A}
    \]
    \[
    \begin{align*}
        \frac{d}{d \lambda } f( \lambda )|_{ \lambda = 0 } 
	&= e^{\lambda A}[A, B] e^{- \lambda A} \\
	&= [A, B] 
    \end{align*}
    \]
    \[
    \begin{align*}
        \frac{d^2 }{d \lambda^2 }f( \lambda ) |_{\lambda =0}
	&= [A, [A,B]] 
    \end{align*}
    \]
    therefore 
    \[
	\begin{align*}
	    e^A B e^{-A} &= \sum_{n=0}^{\infty} \frac{1}{n!} f^{(n)}( \lambda =0) \\
			 &= \sum_{n=0}^{\infty} \frac{1}{n!} [C_A]^n B \\
			 &= e^{C_A} B
	\end{align*}
    \]
    \qed
\end{sol}

\begin{problem}[Homework]
    证明: $e^A B e^A = e^{S_A} B$, where $S_A = \{A, B\}$
\end{problem}

\begin{theorem}[C-B-H equation]
    if $[[A,B],A] = [[A,B],B]$, \text{then} 
    $e^A e^B = e^{A+B} e^{\frac{1}{2}[A,B]}$. 
\end{theorem}
\begin{proof}
    Define $F(t) = e^{At} e^{Bt}$. So 
    \[
    \begin{align*}
        \frac{d}{dt} F(t)
	&= e^{At} (A+B) e^{Bt} \\
	&= [A + e^{At} B e^{-At} ] e^{At} e^{Bt} \\
	&= [A + e^{C_A}B ] e^{A} e^{B} \text{ when t=1} 
    \end{align*}
    \]
\end{proof}

\begin{problem}[Homework]
    证明 $e^{At}e^{Bt} = \exp[ (A+B) t + \frac{1}{2} t^2 [A,B] ] $
\end{problem}

\begin{problem}
    计算 $T(a) R T^{-1} (a) \ket{r} = T(a) R \ket{r-a} = r-a$ . 
    
    首先计算 $T(a) R T ^{-1} (a) $
    \[
	\begin{align*}
	    T(a) R T^{-1}(a) &= \exp( -i \vec{k} \cdot \vec{a}) 
	    R \exp(  i \vec{k} \cdot \vec{a}) \\
			     &= R + 
			     [ -i \vec{k} \cdot \vec{a}, \vec{R}] + 
			     \frac{1}{2} [-i \vec{k} \cdot \vec{a}, [- i \vec{k}
			     \cdot \vec{a}, \vec{R} ]] \\ 
			     &= R - a
	\end{align*}
    \]
    对比可知,
     \[
	 -a = [-i \vec{k} \cdot \vec{a} , \vec{R}]
	 + \frac{1}{2} \Bigl[ 
	     -i \vec{k} \cdot \vec{a}, [ -i \vec{k} \cdot a , \vec{R}] \Bigr]
    \]
\end{problem}

这个题目的计算结果可以推出基本的正则对易关系: 
\[
    [R_i , K_j ] = i \delta _{ij}
\]

类似地, 利用 $[T(dx), T(dy)]=0$ 可以推出 $[1 - iK_y dy, 1-iK_x dx]=0$. 这意味着
\[
    [K_i ,K_j ]=0
\]

至此, 我们可以将 $K$ 认为是动量算符, 其是无限小平移操作的生成元. 

\subsubsection{用生成元推导哈密顿方程} 

作用量定义为 $S = S \int_{t_1}^{t_2} L dt$. 已知哈密顿量 
\[
H = \sum_{\alpha }^{\infty} \frac{\partial L}{\partial \dot{q}_\alpha} - L  
\] 

则哈密顿原理写作
\[
    \begin{align*}
        \delta \int_{t_1}^{t_2} Ldt 
	&= \int_{t_1}^{t_2} \delta \biggl[ \sum_{\alpha}^{\infty} 
	p_\alpha \dot{q}_\alpha - H \biggr] dt \\
	&= 0 \\ 
	\therefore &\int_{t_1}^{t_2} \Bigl[ \sum_{\alpha}^{\infty} 
	\dot{q}_\alpha \delta p_\alpha + p_\alpha \delta \dot{q}_\alpha 
    - \delta H \Bigr] dt =  0 \\ 
    \end{align*}
\] 
\[
\Rightarrow \int_{t_1}^{t_2} \Bigl[ \sum_{\alpha}^{\infty}
(\dot{q}_\alpha \delta p_\alpha - \dot{p}_\alpha \delta q_\alpha
+ \sum_{\alpha}^{\infty} \frac{d}{dt} (p_\alpha \delta q_\alpha ) - \delta H
\Bigr] dt =0
\] 
\[
\Rightarrow  \int_{t_1}^{t_2} \Bigl[ \sum_{\alpha}^{\infty} (\dot{q}_\alpha
\delta p_\alpha - \dot{p}_\alpha \delta q_\alpha - \delta H \Bigr]  dt 
=0
\] 
\[
\Rightarrow  \int_{t_1}^{t_2} \Bigl[ 
\sum_{\alpha}^{\infty}(\dot{q}_\alpha \delta p_\alpha - \dot{p}_\alpha \delta
q_\alpha ) 
- \sum_{\alpha}^{\infty} \Bigl( 
\frac{\partial H}{\partial p_\alpha} \delta p_\alpha + 
\frac{\partial H}{\partial q_\alpha} \delta q_\alpha 
\Bigr) 
\Bigr] dt 
\] 
\[
\Rightarrow \int_{t_1}^{t_2} 
\sum_{\alpha}^{\infty} \biggl[ 
\Bigl( \dot{q}_\alpha - \frac{\partial H}{\partial p_\alpha}  \Bigr) 
- 
\Bigl( \dot{p}_\alpha + \frac{\partial H}{\partial q_\alpha}  \Bigr)
\biggr]  dt 
=0
\] 
因为 $\delta p_\alpha, \delta q_\alpha$ 是独立的变分, 因此
\[
    \begin{align*}
        \frac{\partial H}{\partial p_\alpha} &= \dot{q}_\alpha \\
	\frac{\partial H}{\partial q_\alpha} &= - \dot{p}_\alpha
    \end{align*}
\] 
这就是 {\kaishu 哈密顿正则方程} 

\subsubsection{正则变换} 

我们可以取不同的广义坐标和广义动量作为哈密顿量的变量. 那么要保持哈密顿正则方程
不变, 就需要一个特殊的变换 ~--- 正则变换. 

记新坐标, 动量为 $q_\alpha', p_\alpha'$. 相应的新哈密顿量为 $H'(q'_\alpha,
p'_\alpha$. 

要使得新哈密顿依旧保持哈密顿正则方程, 等价于使新哈密顿量满足最小作用量原理. 
 \[
\begin{align*}
    \delta S' &= \delta \int \Bigl[ L'(q'_\alpha, p'_\alpha) dt \Bigr] \\ 
    &= \delta \biggl[ \int 
    \bigl(\sum_{\alpha}^{\infty}p'_\alpha \dot{q}'_\alpha - H'\bigr) dt \biggr] \\ 
    &= \delta \biggl( \int \sum_{\alpha}^{\infty}
    p_\alpha' dq_\alpha' - H' dt \biggr) \\ 
    &= 0 \\ 
\end{align*}
\] 

原来的哈密顿量肯定满足最小作用量原理, 因此
\[
\delta \biggl( 
\sum_{\alpha}^{\infty}p_\alpha dq_\alpha - Hdt  \biggr) =0	    
\] 

两个式子相加, 得到 
\[
\delta \biggl[ 
\sum_{\alpha}^{\infty} (p_\alpha dq_\alpha - p'_\alpha dq'_\alpha ) 
+ (H-H') dt \biggr] \equiv dU_1 = 0
\] 

观察可知, $\frac{\partial U_1}{\partial q_\alpha} = p_\alpha, 
\frac{\partial U_1}{\partial q'_\alpha} = -p'_\alpha, 
\frac{\partial U_1}{\partial t} = H-H'$ 
这样 $U_1$ 其实是 $q'_\alpha, p'_\alpha$ 的函数, 因此 $U_1$ 就可以描述正则变换. 

通过Legendre 变换可以得到其他三种正则变换母函数. 

\[
\begin{align*}
    dU_1 &= \sum_{\alpha}^{\infty} (p_\alpha dq_\alpha - p'_\alpha dq'_ \alpha) 
    + (H-H') dt	\\ 
    &= \sum_{\alpha }^{\infty} p_ \alpha dq_ \alpha - [ d(p'_ \alpha q'_ \alpha
    ) - q'_ \alpha dp'_ \alpha ] \\ 
    &= \sum_{\sigma }^{\infty} (p_ \alpha dq_ \alpha + q'_ \alpha dp'_ \alpha )  
    - \sum_{\sigma }^{\infty} d(p'_ \alpha q'_ \alpha ) 
    - (H' -H)dt	\\ 
\end{align*}
\] 
因此, 
\[
dU_2 = d(U_1 + \sum_{\alpha }^{\infty} p'_ \alpha q'_ \alpha ) 
\] 

此时, 哈密顿方程依旧不变. 

在把 Legendre 变换的变量改为 $q_ \alpha , q_ \alpha $, 可以得到
\[
U_3 = U_1 - \sum_{\alpha }^{\infty} p_ \alpha q_ \alpha  
\] 

把变量改为 $q_ \alpha , p_ \alpha , q'_ \alpha , p'_ \alpha $, 可得到
\[
U_4 = U_1 - \sum_{\alpha }^{\infty} p_ \alpha q_ \alpha +  
\sum_{\alpha }^{\infty} p'_ \alpha q'_ \alpha  
\] 
\footnote{$U_4$ 其实就是结合 $U_2, u_3$.}
\end{document}

